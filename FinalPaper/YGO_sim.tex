\subsection{Yu-Gi-Oh Simulations}
The Yu-Gi-Oh games are simulated in Python by constructing rules, mentioned in the previous section, that the players must follow and marking points of interest as decision points. Decision points are binary moments in the game where a player can chose to take an action in a given circumstance or do nothing instead. One example would be if a player wants to attack when the opponent has a monster on their field and no spells or traps as well. The action would be to attack or not and the circumstances is the cards the opponent has.

The decision points are given weights, so players can pick with uneven probability. Each combination of weights for all of the decision points would be a possible strategy for a player to use. For instance, a player can choose the strategy of randomly picking with equal probability at each decision or another strategy is for a player to only take an action when the opponent has less monster. 

Taking into consideration all possible actions will make the state space too large and is computationally NP hard. Instead only a handful of decisions points are considered with the rest being concentrated out---given a fixed weight that the players cannot control. Certain decision points are fixed because, base on the rules of the game, deviating from them offers no possible reward. Consider a player deciding to summon a monster to his field. Monsters are the only way to win the game and can defend players from opposing player's monsters when necessary, so there are only advantages to having monsters on your field. Therefore, this decision was fixed so player always summon monsters if able.

The data collected from these simulations is the outcome of the game---win (1) or loose (0), binary variables representing whether a possible strategy---combination of weights---was used by each player or not, and which player went first.

Interestingly, the player who goes first is more likely to win, which is expected, however, as the game progresses the advantage that player has diminishes. In one setting, the game starts the players with no cards in hand and the players loose after taking damage once; this make the games shorter and the ``extra'' card the turn player gets by going first can make a bigger difference. For example, if the going first player draws monster cards twice in a row at the beginning, then they win no matter what the other player does. When players start with more cards in hand and have more life points the advantage of going first should shrink.




