
\subsection{Rules}
In a normal game of Yu-Gi-Oh, there are two players with separate forty card decks that they construct from the available card pool. At the beginning of the game, the players draw five cards from their decks, which are their hands. A player has a deck of cards; his hand---drawn from the deck; his field---cards played from the hand; and his graveyard---cards that were used and no longer in play. Players will decide who gets to go first through a random event---usually by rolling a die. At the start of the game each player has eight thousand life points. The main goal of the game is to reduce the opponent's life points to zero, thus winning the game, although a player can loose by running out of cards in their deck also known as decking out. Three main types of cards exist in the game: (1) monsters, (2) spells, and (3) traps. 

Monsters are used for attacking the opposing player's monsters and life points. Every monster has attack points---a whole number typically between zero and three thousand; in the simplified simulation all monsters will be the same with one thousand attack points. Monsters remain on the field unless destroyed by an opponent's monster or trap.

Spells are utility cards, they have a number of different abilities and uses, but for simplicity I shall limit them to one card. The chosen spell card can be used, from the hand or field, to destroy an opposing player's facedown spell or trap. During their turn, players can use their spells (from hand and field), keep them in hand, or set spells facedown on the field. Since the opposing player does not know if the card set on the field is a spell or trap it can be used to bluff the opponent into thinking it is a trap. Spells go to the graveyard immediately after use.  

Traps are disruption and defensive cards, they have a number of different abilities and uses, but for simplicity I shall limit them to one card. The chosen trap card can be used, from the field, to destroy an opposing player's attacking monster. Players can set traps on their turn, but can only use traps on the opposing player's turn. Traps, like spells, go to the graveyard immediately after use. 

In summary, monsters beat other monsters and damage an opposing player's life points; spells beat other spells or traps; and traps beat monsters.

Players take turns using the cards they have drawn. Each turn follows the order: Draw Phase, Main Phase 1, Battle Phase, Main Phase 2, and End Phase. I will be excluding the last two from my simulations for simplicity.

The draw phase is when the turn player draws a card from their deck, however, if a player has no cards left to draw they loose (decking out). 

Main phase 1 is when the turn player can use their spell cards from hand or field, place their spells or traps from hand to the field facedown, and summon (put on the field) monsters. Only the turn player can summon monsters and they can only summon a monster once during their turn. 

The battle phase is when the turn player uses their monsters to attack their opponent's monsters, except on the first turn. If the opponent has no monsters then the turn player can attack the opponent's life points instead by subtracting the monster's attack points from the player's life points, this is called attacking directly. When a monster attacks an opponent's monster the one with lower attack points is destroyed and placed in the graveyard. In the case of a tie, both monsters are destroyed. The difference in attack points is subtracted from the loosing player's life points. This is also when the opposing player can use their traps on field to destroy the turn player's attacking monsters.

After the turn ends, the cycle restarts with the next player and it continues until one or both of the players have lost all of their life points or decks out.

\subsection{Game Scenario}
For this example, the game will be simplified. Both players will have an infinitely sized deck with equal proportions of monsters, spells, and traps. They will start with no cards in hand, but still draw one card during their draw phase. The players will both only have 1 life point, so whoever successfully attacks directly wins.

Player A goes first, draws a card for turn and sets it facedown. 

Player B draws a monster for turn.

Now, what would player B want to do with this monster? Player B can choose to summon it or keep it for later and if player B summons it they can attack or leave it.

\subsection{Strategy}
Players should always choose to summon a monster if they can. While on the field monsters can defend their players' life points from opposing monsters and, as long as they do not attack, cannot be destroyed by traps. Since players can only summon one monster per their turn stock piling monsters in the hand is pointless, but having them on the field means more monsters to attack the opponents' life points with. Therefore, keeping it in hand should not happen.

To attack or to not attack? Since player B will summon their monster should they attack and risk loosing their only defense to a possible trap? If player B attacks one of two things can happen: player A's facedown is a trap and player B's monster is destroyed or the facedown is a spell and player B will win the game.

If player B attacks, then there is a fifty percent chance that the facedown card player A set is a trap and if it is, there is a one-third chance of player A drawing a monster and winning in the next turn. However, there is a fifty percent chance that the facedown is a bluff, a spell that cannot affect a monster, and player B can win by attacking directly for player A's last life point.

If player B does not attack, then player A will draw a card, take some action, and the cycle will continue until an attack is made. Since Player A has the advantage of going first, meaning that the player will always have one extra card on their turn, this cycle will favor the going first player.

In this simplified setting, the best response can be roughly solved for by comparing the expected values of each option. Outside of this setting, however, the game can be much too complex to solve normally. By simulating some of the possible outcomes of the game with different strategies and using a logistic regression model an approximation of the probability of winning can be made for each strategy. 

In this example, the strategy is a mixed-strategy of attack or do not attack. This decision is split into different groups based on what each player has on their field, for example, having a facedown versus just a monster. I shall go into more detail on the model later in this paper, but the results come from the formula below,
$$ \text{Pr(Player B wins)} = \frac{\exp(\beta_0 + \beta_1 x_1 + ... + \beta_K x_K)}{1 + \exp(\beta_0 + \beta_1 x_1 + ... + \beta_K x_K)}, \quad k=1,...K , $$
where $\beta_0, \beta_1, ..., \beta_K$ are the regression coefficients and $x_1,..., x_K$ are the mixed strategies. With the the strategy leaning towards attacking having the highest estimated probability of winning. Player B will want to risk the trap card to take advantage of player A's weakness before they can utilize their card advantage.



