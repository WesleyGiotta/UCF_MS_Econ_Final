
\section{Matching Pennies}

\lstinputlisting[language=python]{../MatchingPennies/original_MP.py}

The matching pennies script is meant to show how players can evolve over time and can converge to an optimal solution. The first two functions are the players and their strategy is how much weight they give to the choices (heads or tails). The next function adjusts the loosing player's strategy by increments of 0.05 based on the history of the last 20 outcomes. The game function keeps track of the score and calls the weight adjusting function every 20 turns. After about 1000 simulations it settles down, with some error, to about 0.5.

\FloatBarrier
\begin{table}[]
\caption{Matching Pennies (First 10 Simulations)}
\begin{tabular}{lllllll}
\hline
A & B & ScoreA & ScoreB & Turns & WeightA\_heads & WeightB\_heads \\
\hline
T & H & 0.0    & 1.0    & 1.0   & 0.0            & 1.0            \\
T & H & 0.0    & 2.0    & 2.0   & 0.0            & 1.0            \\
T & H & 0.0    & 3.0    & 3.0   & 0.0            & 1.0            \\
T & H & 0.0    & 4.0    & 4.0   & 0.0            & 1.0            \\
T & H & 0.0    & 5.0    & 5.0   & 0.0            & 1.0            \\
T & H & 0.0    & 6.0    & 6.0   & 0.0            & 1.0            \\
T & H & 0.0    & 7.0    & 7.0   & 0.0            & 1.0            \\
T & H & 0.0    & 8.0    & 8.0   & 0.0            & 1.0            \\
T & H & 0.0    & 9.0    & 9.0   & 0.0            & 1.0            \\
T & H & 0.0    & 10.0   & 10.0  & 0.0            & 1.0           \\
\hline
\end{tabular}
\end{table}


\FloatBarrier
\section{Rock-Paper-Scissors}

\begin{lstlisting}
import random
import pandas as pd

def PlayerA(weightA, n):
    Player_A = random.choices(population=['Rock','Paper', 'Scissors'],weights=weightA, k=n)
    return Player_A
    
def PlayerB(weightB, n):
    Player_B = random.choices(population=['Rock','Paper', 'Scissors'],weights=weightB, k=n)
    return Player_B

def Game(x, wA, wB):
    scoreA = 0
    scoreB = 0
    turns = x
    history = pd.DataFrame() 
    A = PlayerA(wA, x)
    B = PlayerB(wB, x)
    
    for play in range(x):
        if A[play] == B[play]:
            pass
        elif A[play] == 'Rock':
            if B[play] == 'Paper':
                scoreB += 1
            else:
                scoreA += 1    
        elif A[play] == 'Paper':
            if B[play] == 'Scissors':
                scoreB += 1
            else:
                scoreA += 1
        else:
            if B[play] == 'Rock':
                scoreB += 1
            else:
                scoreA += 1
            
    history = history.append({'Turns':turns, 'ScoreA':scoreA, 'ScoreB':scoreB,
                              'wA_Rock':wA[0], 'wA_Paper':wA[1], 'wA_Scissors':wA[2],
                              'wB_Rock':wB[0], 'wB_Paper':wB[1], 'wB_Scissors':wB[2],
                              'weights_A':str(wA), 'weights_B':str(wB)}, ignore_index=True)
                    
    return history

fixed = pd.DataFrame()
for play in range(0,1000):
    fixed = fixed.append(Game(1000, [0.2, 0.6, 0.2], [0.4, 0.5, 0.1]))
\end{lstlisting}

The rock-paper-scissors script plays out multiple games at fixed strategies and then I can compare the distributions of the players payoffs to look for a combination of strategies where the payoffs overlap. The first two functions are the players and their strategy is how much weight they give to the choices (rock, paper, or scissors). It is slightly different from matching pennies because instead of a single game the functions are set to pick a vector of n games at a time. The game function then keeps track of the score for each n game and records the total. I run a thousand of these vector games at a time for a fixed weight and create a distribution of each player's score. 

\FloatBarrier
\begin{table}[]
\caption{Rock-Paper-Scissors (First 10 Simulations)}
\begin{tabular}{lllll}
\hline
ScoreA & ScoreB & Turns  & weights\_A          & weights\_B          \\
\hline
335.0  & 252.0  & 1000.0 & {[}0.2, 0.6, 0.2{]} & {[}0.4, 0.5, 0.1{]} \\
363.0  & 232.0  & 1000.0 & {[}0.2, 0.6, 0.2{]} & {[}0.4, 0.5, 0.1{]} \\
394.0  & 223.0  & 1000.0 & {[}0.2, 0.6, 0.2{]} & {[}0.4, 0.5, 0.1{]} \\
344.0  & 255.0  & 1000.0 & {[}0.2, 0.6, 0.2{]} & {[}0.4, 0.5, 0.1{]} \\
358.0  & 242.0  & 1000.0 & {[}0.2, 0.6, 0.2{]} & {[}0.4, 0.5, 0.1{]} \\
363.0  & 253.0  & 1000.0 & {[}0.2, 0.6, 0.2{]} & {[}0.4, 0.5, 0.1{]} \\
356.0  & 233.0  & 1000.0 & {[}0.2, 0.6, 0.2{]} & {[}0.4, 0.5, 0.1{]} \\
367.0  & 229.0  & 1000.0 & {[}0.2, 0.6, 0.2{]} & {[}0.4, 0.5, 0.1{]} \\
362.0  & 232.0  & 1000.0 & {[}0.2, 0.6, 0.2{]} & {[}0.4, 0.5, 0.1{]} \\
365.0  & 232.0  & 1000.0 & {[}0.2, 0.6, 0.2{]} & {[}0.4, 0.5, 0.1{]} \\
\hline
\end{tabular}
\end{table}

\FloatBarrier
\section{Yu-Gi-Oh}

\lstinputlisting[language=python]{../Game/rules_hand_field.py}
\lstinputlisting[language=python]{../Game/mechanics_cs.py}
\lstinputlisting[language=python]{../Game/gameplay_turns.py}
\lstinputlisting[language=python]{../Game/max_weightFunc_2.py}
\lstinputlisting[language=python]{../Game/currnent_state.py}


The Yu-Gi-Oh scripts plays a single game at a time and outputs the strategy used along with the weights for each decision point. In table 5, the columns are FirstA---if player A went first; HA---how many of each card type was in player A's hand; LP\_A---if player A won or not; PrAw---the weights for each decision point; and As/Bs---if a strategy was used by a player. 

The scripts follow the rules as highlighted in section 3 and the script will loop until a win condition is met. The script is broken up into general rules, phases of a turn, and a full turn for both players.


\FloatBarrier
{\footnotesize
\begin{table}[]
\caption{Yu-Gi-Oh cards in hand and attack weights}
\begin{tabular}{rrrrrrrr}
\toprule
 FirstA &  HAMon &  HASpell &  HATrap &  LP\_A &  PrAwM &  PrAwMST &  PrAwST \\
\midrule
    0.0 &    2.0 &      2.0 &     1.0 &   0.0 &    1.0 &      1.0 &     1.0 \\
    1.0 &    2.0 &      0.0 &     0.0 &   1.0 &    1.0 &      1.0 &     1.0 \\
    1.0 &    2.0 &      0.0 &     0.0 &   0.0 &    1.0 &      1.0 &     1.0 \\
    0.0 &    2.0 &      0.0 &     3.0 &   1.0 &    1.0 &      1.0 &     1.0 \\
    0.0 &    2.0 &      1.0 &     2.0 &   0.0 &    1.0 &      1.0 &     1.0 \\
    0.0 &    1.0 &      1.0 &     3.0 &   0.0 &    1.0 &      1.0 &     1.0 \\
    1.0 &    1.0 &      0.0 &     0.0 &   1.0 &    1.0 &      1.0 &     1.0 \\
    1.0 &    3.0 &      0.0 &     0.0 &   1.0 &    1.0 &      1.0 &     1.0 \\
    1.0 &    2.0 &      0.0 &     0.0 &   0.0 &    1.0 &      1.0 &     1.0 \\
    0.0 &    3.0 &      1.0 &     1.0 &   1.0 &    1.0 &      1.0 &     1.0 \\
\bottomrule
\end{tabular}

\end{table}
}
{
\begin{table}[]
\caption{Yu-Gi-Oh strategy used and attack weights}
\scriptsize
\begin{tabular}{rrrrrrrrrrr}
\toprule
 AsADP &  AsAlways &  AsHalf &  BsAlways &  BsHalf &  FirstA &  LP\_A &  LP\_B &  PrAwM &  PrAwMST &  PrAwST \\
\midrule
   0.0 &       1.0 &     0.0 &       1.0 &     0.0 &     0.0 &   0.0 &   1.0 &    1.0 &      1.0 &     1.0 \\
   0.0 &       1.0 &     0.0 &       1.0 &     0.0 &     0.0 &   0.0 &   1.0 &    1.0 &      1.0 &     1.0 \\
   0.0 &       1.0 &     0.0 &       1.0 &     0.0 &     1.0 &   1.0 &   0.0 &    1.0 &      1.0 &     1.0 \\
   0.0 &       1.0 &     0.0 &       1.0 &     0.0 &     1.0 &   1.0 &   0.0 &    1.0 &      1.0 &     1.0 \\
   0.0 &       1.0 &     0.0 &       1.0 &     0.0 &     1.0 &   1.0 &   0.0 &    1.0 &      1.0 &     1.0 \\
   0.0 &       1.0 &     0.0 &       1.0 &     0.0 &     0.0 &   0.0 &   1.0 &    1.0 &      1.0 &     1.0 \\
   0.0 &       1.0 &     0.0 &       1.0 &     0.0 &     1.0 &   1.0 &   0.0 &    1.0 &      1.0 &     1.0 \\
   0.0 &       1.0 &     0.0 &       1.0 &     0.0 &     0.0 &   0.0 &   1.0 &    1.0 &      1.0 &     1.0 \\
   0.0 &       1.0 &     0.0 &       1.0 &     0.0 &     1.0 &   1.0 &   0.0 &    1.0 &      1.0 &     1.0 \\
   0.0 &       1.0 &     0.0 &       1.0 &     0.0 &     0.0 &   0.0 &   1.0 &    1.0 &      1.0 &     1.0 \\
\bottomrule
\end{tabular}

\end{table}
}



